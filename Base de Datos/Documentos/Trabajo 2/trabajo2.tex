\documentclass[a4paper,12pt]{article}
\usepackage[utf8]{inputenc}
\usepackage[spanish]{babel}
\usepackage{color}
\usepackage{parskip}
\usepackage{graphicx}
\usepackage{multirow}
\usepackage{listings}
\definecolor{mygreen}{rgb}{0,0.6,0}
\definecolor{lbcolor}{rgb}{0.9,0.9,0.9}

\lstset{
backgroundcolor=\color{lbcolor},
    tabsize=4,    
%   rulecolor=,
    language=SQL,
        basicstyle=\scriptsize,
        aboveskip={1.5\baselineskip},
        columns=fixed,
        showstringspaces=false,
        extendedchars=false,
        breaklines=true,
        prebreak = \raisebox{0ex}[0ex][0ex]{\ensuremath{\hookleftarrow}},
        frame=single,
        numbers=left,
        showtabs=false,
        showspaces=false,
        showstringspaces=false,
        identifierstyle=\ttfamily,
        keywordstyle=\color[rgb]{0,0,1},
        commentstyle=\color[rgb]{0.026,0.112,0.095},
        stringstyle=\color{red},
        numberstyle=\color[rgb]{0.205, 0.142, 0.73},
%        \lstdefinestyle{C++}{language=C++,style=numbers}’.
}

\begin{document}

\section{Ejercicio 1}

\begin{lstlisting}
create database tienda;
use tienda;

create table Clientes(
DNI char(8) not null,
apellidos varchar(100) not null,
nombres varchar (100) not null,
dir varchar(200) not null,
fechanac date not null);

describe Clientes;

alter table Clientes
add constraint pk_Clientes
primary key(DNI);

create table Producto(
codPro int not null,
nombre varchar(100) not null,
RUC char(11) not null);

describe Producto;

alter table Producto
add constraint pk_Producto
primary key(codPro);

create table Compras(
codCompras int not null,
DNI char(8) not null,
codPro int not null);

alter table Compras
add constraint pk_compras
primary key(codCompras);

alter table Compras
add constraint fk_compras_cliente
foreign key(DNI)
references Clientes(DNI);

alter table Compras
add constraint fk_compras_producto
foreign key(codPro)
references Producto(codPro);

create table Proveedor(
RUC char(11) not null,
nombre varchar(100) not null,
dir varchar(200));

describe Proveedor;

alter table Proveedor
add constraint pk_proveedor
primary key(RUC);

alter table Producto
add constraint fk_producto_proveedor
foreign key(RUC)
references Proveedor(RUC);
\end{lstlisting}

\section{Ejercicio 2}

\begin{lstlisting}
create database ejercicio2;
use ejercicio2;

create table Camiones(
Matricula char(11) not null,
Modelo varchar(100) not null,
Tipo varchar(100) not null,
Potencia varchar(100) not null);

alter table Camiones
add constraint pk_camiones
primary key(Matricula);

describe Camiones;

create table Camiones_Camioneros(
Cod int not null,
Matricula char(11) not null,
DNI char(8) not null);

alter table Camiones_Camioneros
add constraint pk_Camiones_Camioneros
primary key(Cod);

describe Camiones_Camioneros;

create table Camioneros(
DNI char(8) not null,
Ciudad varchar(100),
Nombre varchar(100) not null,
Telefono varchar(20),
Direccion varchar(200),
Salario int not null);

alter table Camioneros
add constraint pk_Camioneros
primary key(DNI);

describe Camioneros;

alter table Camiones_Camioneros
add constraint fk_Camiones_CC
foreign key(Matricula)
references Camiones(Matricula);

alter table Camiones_Camioneros
add constraint fk_Camioneros_CC
foreign key(DNI) 
references Camioneros(DNI);

create table Paquetes(
CodPaq int not null,
Descripcion varchar(200) not null,
Destinatario varchar(100) not null,
Dir varchar(200) not null,
DNI char(8) not null,
CodProv int not null);


alter table Paquetes
add constraint pk_paquetes
primary key(CodPaq);

describe Paquetes;

create table Provincia(
CodProv int not null,
Nombre varchar(100));

alter table Provincia
add constraint pk_provincia
primary key(CodProv);

describe Provincia;

alter table Paquetes
add constraint fk_paquetes_provincia
foreign key(CodProv)
references Provincia(CodProv);

alter table Paquetes
add constraint fk_paquetes_camioneros
foreign key(DNI)
references Camioneros(DNI);
\end{lstlisting}

\section{Ejercicio 3}

\begin{lstlisting}
create database ejercicio3;
use ejercicio3;

create table Profesores(
DNI char(8) not null,
Nombre varchar(100) not null,
Direccion varchar(200),
Telefono varchar(20));

alter table Profesores
add constraint pk_profesores
primary key(DNI);

describe Profesores;

create table Modulos(
CodMod int not null,
Nombre varchar(100) not null,
DNI char(8) not null);

alter table Modulos
add constraint pk_modulos
primary key(CodMod);

describe Modulos;

create table Alumno(
Expediente int not null,
Nombre varchar(100) not null,
FechaNac date,
Apellidos varchar(100) not null);

alter table Alumno
add constraint pk_alumno
primary key(Expediente);

describe Alumno;

create table Alumno_Modulo(
Cod int not null,
Expediente int not null,
CodMod int not null,
Delegado boolean not null);

alter table Alumno_Modulo
add constraint pk_AM
primary key(Cod);

alter table Alumno_Modulo
add constraint fk_AM_Modulos
foreign key(CodMod)
references Modulos(CodMod);

alter table Alumno_Modulo
add constraint fk_AM_Alumno
foreign key(Expediente)
references Alumno(Expediente);

alter table Modulos
add constraint fk_Modulos_Profesores
foreign key(DNI)
references Profesores(DNI);
\end{lstlisting}

\section{Ejercicio 4}

\begin{lstlisting}
create database ejercicio4;
use ejercicio4;


create table Coche(
Matricula char(11) not null,
Marca varchar(100) not null,
Precio int not null,
Color varchar(100),
CodEmpresa int not null,
Modelo varchar(100));

alter table Coche
add constraint pk_coche
primary key(Matricula);

describe Coche;

create table Revision(
CodRev int not null,
Matricula char(11) not null);

alter table Revision
add constraint pk_revision
primary key(CodRev);

create table Cliente(
CodEmpresa int not null,
DNI char(8) not null,
Nombre varchar(100) not null,
Dir varchar(200),
Ciudad varchar(100),
Telefono char(8));

alter table Cliente
add constraint pk_Clientes
primary key(CodEmpresa);

create table Partes(
CodParte int not null,
Descripcion varchar(200) not null);

alter table Partes
add constraint pk_Partes
primary key(CodParte);

create table Partes_Revision(
CodPR int not null,
CodParte int not null,
CodRev int not null);

alter table Partes_Revision
add constraint pk_PR
primary key(CodPR);

alter table Coche
add constraint fk_Coche_Cliente
foreign key(CodEmpresa)
references Cliente(CodEmpresa);

alter table Revision
add constraint fk_Revision_Coche
foreign key(Matricula)
references Coche(Matricula);

alter table Partes_Revision
add constraint fk_PR_Revision
foreign key(CodRev)
references Revision(CodRev);

alter table Partes_Revision
add constraint fk_PR_Partes
foreign key(CodParte) 
references Partes(CodParte);
\end{lstlisting}

\section{Ejercicio 5}

\begin{lstlisting}
create database ejercicio5;
use ejercicio5;

create table Paciente(
CodPaciente int not null,
Nombre varchar(100) not null,
Apellido varchar(100) not null,
Dir varchar(200),
Ciudad varchar(100),
Distrito varchar(100),
Telefono char(8),
FechaNac date);

alter table Paciente
add constraint pk_Paciente
primary key(CodPaciente);

describe Paciente;

create table Ingreso(
CodIngreso int not null,
Cama varchar(20) not null,
Habitacion varchar(20) not null,
Fecha date not null,
CodPaciente int not null,
CodMedico int not null);

alter table Ingreso
add constraint pk_Ingreso
primary key(CodIngreso);

describe Ingreso;

create table Medico(
CodMedico int not null,
Nombre varchar(100) not null,
Apellidos varchar(100) not null,
Especialidad varchar(100) not null,
Telefono char(8));

alter table Medico
add constraint pk_Medico
primary key(CodMedico);

describe Medico;

alter table Ingreso
add constraint fk_Ingreso_Medico
foreign key(CodMedico)
references Medico(CodMedico);

alter table Ingreso
add constraint fk_Ingreso_Paciente
foreign key(CodPaciente)
references Paciente(CodPaciente);
\end{lstlisting}

\end{document}