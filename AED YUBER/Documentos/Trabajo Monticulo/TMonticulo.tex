\documentclass[a4paper,12pt]{article}
\usepackage[utf8]{inputenc}
\usepackage[spanish]{babel}
\usepackage{color}
\usepackage{parskip}
\usepackage{graphicx}
\usepackage{multirow}
\usepackage{listings}
\definecolor{mygreen}{rgb}{0,0.6,0}
\definecolor{lbcolor}{rgb}{0.9,0.9,0.9}

\lstset{
backgroundcolor=\color{lbcolor},
    tabsize=4,    
%   rulecolor=,
    language=[GNU]C++,
        basicstyle=\scriptsize,
        aboveskip={1.5\baselineskip},
        columns=fixed,
        showstringspaces=false,
        extendedchars=false,
        breaklines=true,
        prebreak = \raisebox{0ex}[0ex][0ex]{\ensuremath{\hookleftarrow}},
        frame=single,
        numbers=left,
        showtabs=false,
        showspaces=false,
        showstringspaces=false,
        identifierstyle=\ttfamily,
        keywordstyle=\color[rgb]{0,0,1},
        commentstyle=\color[rgb]{0.026,0.112,0.095},
        stringstyle=\color{red},
        numberstyle=\color[rgb]{0.205, 0.142, 0.73},
%        \lstdefinestyle{C++}{language=C++,style=numbers}’.
}

\begin{document}

\section{Monticulo.h}

\begin{lstlisting}

#ifndef MONTICULO_H
#define MONTICULO_H
#include <vector>
#include <iostream>
using namespace std;

class Monticulo
{
    public:
        Monticulo();
        virtual ~Monticulo();
        void ingresar(int);
        void print();
        int size();
        int del();
    protected:
    private:
        vector<int> monti;
};

Monticulo::Monticulo(){}
Monticulo::~Monticulo(){}

int Monticulo::size(){
    return monti.size();
}

int Monticulo::del(){
    int resultado = monti.front();
    if(monti.size() == 0)return resultado;
    monti[0] = monti[monti.size() - 1];
    monti.pop_back();
    int pos = 0;
    while(pos <= monti.size() - 1){
        if(2 * pos + 1 > monti.size() - 1)return resultado;
        if(2 * pos + 2 > monti.size() - 1){
            if(monti[pos] > monti[2 * pos + 2]){
                auto temp = monti[pos];
                monti[pos] = monti[2 * pos + 2];
                monti[2 * pos + 2] = temp;
                return resultado;
            }
        }
        if(monti[pos] > monti[2 * pos + 1] or monti[pos] > monti[2 * pos + 2]){
            if(monti[2 * pos + 1] < monti[2 * pos + 2]){
                auto temp = monti[pos];
                monti[pos] = monti[2 * pos + 1];
                monti[2 * pos + 1] = temp;
                pos = 2 * pos + 1;
            }
            else{
                auto temp = monti[pos];
                monti[pos] = monti[2 * pos + 2];
                monti[2 * pos + 2] = temp;
                pos = 2 * pos + 2;
            }
        }
        else{
            break;
        }
    }
    return resultado;
}

void Monticulo::print(){
    for(int i = 0; i < monti.size(); i++){
        cout<<monti[i]<<endl;
    }
}

void Monticulo::ingresar(int valor){
    monti.push_back(valor);
    int pos = monti.size() - 1;
    while(pos > 0){
        if(monti[pos] < monti[(pos - 1) / 2]){
            auto temp = monti[pos];
            monti[pos] = monti[(pos - 1) / 2];
            monti[(pos - 1) / 2] = temp;
        }
        pos = (pos - 1) / 2;
    }
}

#endif // MONTICULO_H

\end{lstlisting}

\section{Main.cpp}

\begin{lstlisting}
#include <iostream>
#include "Monticulo.h"
#include <fstream>
#include "math.h"
using namespace std;

float convertirNumero(string numero){
    double resultado = 0;
    auto iter = numero.end();
    iter--;
    double contador = 0;
    for(iter; iter!= numero.begin(); iter--){
        if((*iter) == '.'){
            resultado /= pow(10,contador);
            contador = -1;
        }
        else{
            resultado += pow(10,contador) * ((*iter) - 48);
        }
        contador++;
    }
    resultado += pow(10,contador) * ((*iter) - 48);
    return resultado;
}

bool esNumero(string::iterator &letra){
    if(*letra >= 48 and *letra <= 57)return true;
    return false;
}

float verificarLinea(string linea){
    int estado = 0;
    string resultado;
    for(auto iter = linea.begin(); iter != linea.end(); ++iter){
        if(estado == 5)break;
        switch(estado){
            case 0:
                if(*iter == 116)estado = 1;
                else return -1;
                break;
            case 1:
                if(esNumero(iter))estado = 2;
                else return -1;
                break;
            case 2:
                if(esNumero(iter))estado = 2;
                else if(*iter == 32)estado = 3;
                else return -1;
                break;
            case 3:
                if(esNumero(iter)){
                    estado = 4;
                    resultado.insert(resultado.end(),*iter);
                }
                else return -1;
                break;
            case 4:
                if(esNumero(iter)){
                    estado = 4;
                    resultado.insert(resultado.end(),*iter);
                }
                else if(*iter == 32) estado = 5;
                else return -1;
                break;
            default:
                return -1;
        }
    }
    return convertirNumero(resultado);
}

int main()
{
    Monticulo impresion;
    ifstream archivo("hojas.txt");
    if(archivo.fail()){
        cout<<"No se pudo abrir el archivo"<<endl;
        return 0;
    }
    char linea[128];
    int numberLine = 0;
    while(archivo.getline(linea,128)){
        numberLine++;
        string tempLinea(linea);
        auto number = verificarLinea(tempLinea);
        if(number == -1){
            cout<<"Error en la linea "<<numberLine<<endl;
        }
        else{
            impresion.ingresar(number);
        }
    }
    archivo.close();
    auto tam = impresion.size();
    for(int i = 0; i < tam; i++){
        cout<<"Impresion numero "<<i+1<<" :"<<impresion.del()<<" hojas"<<endl;
    }
}





\end{lstlisting}

\end{document}